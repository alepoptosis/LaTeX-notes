\documentclass[12pt, letterpaper]{article} % defines the type (class) of the 
% document, takes [extra parameters, separated by commas] like font (def = 10) 
% and paper size (a3paper, a4paper) (e.g. [12pt, letterpaper])

\usepackage[utf8]{inputenc} % sets encoding for the document

\usepackage{graphicx} % allows use of pictures in document
\graphicspath{{images/}{more images/}} % default directories for images

\usepackage{geometry} % allows to change length & layout various elements
\geometry{a4paper, portrait, margin = 1in} % paper + orientation + margin size

\usepackage{multicol} % allows to have multiple columns
\setlength{\columnsep}{1cm} % sets columns spacing

\usepackage[
backend=biber, % default backend to sort the library, bibtex is a simpler alt
style=apa, % bibliography and citation style
sorting=ynt
]{biblatex} % allows easier bibliography management
\addbibresource{test.bib} % imports bibliography file

\usepackage[
capitalise, % remove for lowercase labels
noabbrev % remove for abbreviated labels
]{cleveref} % produces better labels when referencing

\usepackage[dvipsnames]{xcolor} % allows use of colour in document
\pagecolor{white} % sets page background colour, reversed by \nopagecolor 
\color{black} % sets default text colour

\definecolor{mypink1}{rgb}{0.858, 0.188, 0.478} % four ways to define custom colours
\definecolor{mypink2}{RGB}{219, 48, 122}
\definecolor{mypink3}{cmyk}{0, 0.7808, 0.4429, 0.1412}
\definecolor{mygray}{gray}{0.6}

\colorlet{LightRubineRed}{RubineRed!70!} % three ways to customize colours (xcolor only)
\colorlet{OrangeGreen}{green!10!orange!90!}
\definecolor{HTMLColor}{HTML}{00F9DE}

\setlength{\parindent}{20pt} % sets the size of the indent
\setlength{\parskip}{1ex} % sets the paragraph spacing
\renewcommand{\baselinestretch}{1.15} % sets interline spacing
\renewcommand{\familydefault}{\rmdefault} % sets font family (alt: sf, tt)
\pagenumbering{arabic} % sets the style of page numbering

\title{\LaTeX notes} % title of the document

% author of the document + affiliations
\author{alepoptosis\thanks{find me on twitter!}, too much tea} 

\date{April 2019} % date, can take the command \today

\begin{document} % start of the document body

\maketitle % inserts title + author + date in the document body

\section{Font style and size}

\subsection*{Here is a series of font styles:}

Use \verb+\textbf{}+ \textbf{for bold!} \\
Use \verb+\textit{}+ \textit{for italics!} \\
Use \verb+\underline{}+ \underline{to underline!} \\
Also \textbf{\textit{\underline{combine them!}}} \\
Use \verb+\textsl{}+ \textsl{for slanted!} \\
Use \verb+\textsc{}+ \textsc{for small caps!} \\
Use \verb+\emph{}+ \emph{emphasize!} % can be altered by packages

\subsection*{Here are the default font families:}

\verb+\textrm{}+ \textrm{makes text Serif (roman)!} \\
\verb+\textsf{}+ \textsf{makes it Sans Serif!} \\
\verb+\texttt{}+ \texttt{makes it Typewriter (monospace)!} 

\subsection*{Here are some font sizes (relative to starting size):} 
% list not exhaustive
Here, size names are the commands (case sensitive)! e.g. 
\verb+\tiny{}, \large{}, \Large{} and \huge{}+\\

\tiny{This is tiny!}
\small{This is small!}
\large{This is large!}
\Large{I said Large!}
\LARGE{NO, LARGE!!!}
\huge{Now huge!}
\normalsize{Aaand back to normal size.}

\clearpage % inserts page break, use before the start of a new section
% \newpage is an alt that breaks the page exactly at that point + starts a new
% column (if using more than one) - \clearpage is usually preferred

\section{Formatting}

\begin{abstract}
	\verb+\begin{abstract}+ formats an abstract environment.
\end{abstract}

This is a first paragraph.\\
\verb+\\+ creates a line break.\newline
\verb+\newline+ also creates a line break. % list not exhaustive

\begin{center}
	\verb+\begin{center}+ centres the text! You can also align it left (\verb+\flushleft+) or right (\verb+\flushright+), or do nothing 
	to leave it justified.
	
	A blank line creates a new paragraph!\par
	The \verb+\par+ command also creates a new paragraph!
\end{center}

You can insert horizontal space in text! \\
You can set a specific width with \verb+\hspace{width}+ \hspace{1cm} (all units 
allowed.)
or you can have a break with \verb+\hfill+ that automatically \hfill 
fills all the space available. \\ 
You can make this pretty by adding a line with \verb+\hrulefill+ \hrulefill \\
or dots with \verb+\dotfill+ \dotfill. Useful for signatures and indices!

You can also insert vertical space! 
Again, you can either use \verb+\vspace{height}+ to \vspace{5mm} \\
specify a distance or use \verb+\vfill+ to automatically \vfill fill the space 
available 
(will take into account other elements in page). \\  
\verb+\smallskip, \medskip and \bigskip+ are also 
somewhat dynamic ways to insert vertical space.

\begin{multicols}{3} % call multicols* for unbalanced columns
	[\subsection*{Columns}
	Square brackets mark the header text on top of the columns. 
	Put whatever you want here but figures/tables.]
	This text should end up in column 1. \\ 
	This text should end up in column 2. \\ 
	This text should end up in column 3.
\end{multicols}
And would you look at that - it did. Yay.


\clearpage
\subsection*{Sections hierarchy (commands):} % see titlesec for customisation

\begin{enumerate}
	\setcounter{enumi}{-2} % start at a different number
	\item part: only available in report doc class
	\item chapter: only available in book doc class
	\item section
	\item subsection
	\item subsubsection
	\item paragraph
	\item subparagraph
\end{enumerate}
Any of these can be made unnumbered by adding an * before the opening curly 
brace.

\subsection*{Units accepted by \LaTeX:}

\begin{table}[!h] % made with tablesgenerator.com (see tables section)
	\centering
	\begin{tabular}{|l|l|}
		\hline
		Abbreviation & Value                                      \\ \hline
		pt           & $\sim$0.0138 inch or 0.3515 mm             \\ \hline
		mm           & a millimeter                               \\ \hline
		cm           & a centimeter                               \\ \hline
		in           & an inch                                    \\ \hline
		ex           & $\sim$height of an 'x' in the current font \\ \hline
		em           & $\sim$width of an 'M' in the current font  \\ \hline
		mu           & math unit equal to 1/18 em                 \\ \hline
	\end{tabular}
	\caption{\LaTeX units summary}
	\label{table:units}
\end{table}

\subsection*{Paragraph formatting}

By default, the first paragraph of a section or a chapter is not indented.

The second is, and the indent size is defined in the preamble.

All subsequent ones are too.

\noindent The \verb+\noindent+ command changes this.\par

This is a paragraph that spans multiple lines that I am using to test the 
line spacing options. Really there is not that much to see here. Carry on.

\clearpage
\section{Figures and referencing} \label{sec:fig} % labels for referencing

\begin{figure}[h] % inserts figure + caption, label and reference [h]ere
    \centering % page alignment
    \includegraphics[width=0.9\textwidth]{scream.jpg} % inserts image
    % with width of 90% that of text
    \caption{me} % inserts caption, used in list of figures
    \label{fig:scream} % numbers and labels image for reference
\end{figure}

\verb+\cref{}+, from the \texttt{cleveref} package, is used to reference the 
figure label, \cref{fig:scream}, while \verb+\cpageref{}+ will refer to the 
page it is in, \cpageref{fig:scream}. You can also use \verb+\cref{}+ to refer 
to a section, like \cref{sec:fig}. 

Change settings for the package in the preamble. You can use the built-in 
reference system with \verb+\ref{} and \pageref{}+, but it will only produce 
the number, and not the label.

\clearpage
\section{Lists and maths}
Environments are sections of the document that present themselves 
in a different way to the rest. They start with \verb+\begin{}+ and end in... 
well, \verb+\end{}+. You already encountered plenty in the previous sections. 
Here are some of them in more details.

\subsection*{Lists}

\begin{itemize}
  \item \verb+\itemize+ environment
  \item This is an unordered list
  \item It uses bullet points
  \item Text can be of any length
\end{itemize}

\begin{enumerate}
% \setcounter{enumi}{3} <- start at a different number
  \item \verb+\enumerate+ environment
  \item This is an ordered list
  \item It uses numbers
  \item The list number increases with each item
\end{enumerate}

\begin{description}
	\item[Description environment:] adds a description for every item!
	\item[This is an item,] and this the description.
	\item[Numbering:] none.
\end{description}

\subsection*{Math}

These are ways to write mathematical expressions, inline mode:
$E=mc^2$, 
\(E=mc^2\), 
\begin{math} E=mc^2 \end{math}

These are ways to write mathematical expressions, display mode:

\begin{equation} \label{eq:1} % numbered
E=mc^2
\end{equation}

\[E=mc^2\] % unnumbered

\begin{displaymath} % unnumbered alt
E=mc^2
\end{displaymath}

% do NOT use $$ E=mc^2 $$ 

Like everything else, display mode mathematical equations can be referenced 
with \verb+\cref+, e.g. \cref{eq:1}. Unnumbered equations can also be 
referenced by number but cannot be easily recognised.

\clearpage

\subsection*{Tables} % use tablesgenerator.com to convert spreadsheets to code

\begin{table}[h] % inserts table [h]ere
	\centering % centers the table in page
	\begin{tabular}{|| l | c | r ||} % defines layout, number/align of columns
		% & vertical lines
		\hline % line at the top of first row
		Col1 &	Col2 &	Col2	\\ \hline\hline % & breaks cell, \\ breaks row
		A	 &	23	 &	10000	\\ \hline
		B	 &	2	 &	66		\\ \hline
		C	 &	5	 &	58 		\\ [1ex] \hline % [ex] adds vertical space
	\end{tabular}
	\caption{Fancy table} % label and caption go after the table
	\label{table:data} % numbers and labels the table for reference
\end{table}

The table can then be referred to as \cref{table:data} on \cpageref{table:data}.

\section{Colours}

The \texttt{xcolour} package allows to colour things. 
The basic colours it supports are 
\textcolor{white}{white}, 
\textcolor{black}{(white), black}, 
\textcolor{red}{red}, 
\textcolor{green}{green}, 
\textcolor{blue}{blue}, 
\textcolor{cyan}{cyan}, 
\textcolor{magenta}{magenta}, 
\textcolor{yellow}{and yellow}.

Adding \texttt{dvipsnames} to the package in the preamble 
allows you to name a few more, like 
\textcolor{TealBlue}{Teal Blue} or 
\textcolor{WildStrawberry}{Wild Strawberry} 
(careful with the caps!). 
You can also change \colorbox{BurntOrange}{the background colour} 
for the text.

The color command (instead of textcolor) can be used 
\color{cyan}to switch the colour of an entire block of text 
until it ends - or until the end of the environment. 
Remember to \color{black} switch back in that case!

Here is a list of custom colours (defined in the preamble) instead.
\begin{itemize}
	\item \textcolor{mypink1}{Pink with rgb}
	\item \textcolor{mypink2}{Pink with RGB}
	\item \textcolor{mypink3}{Pink with cmyk}
	\item \textcolor{mygray}{Gray with gray}
	\item \textcolor{LightRubineRed}{Rubine red at 70\% intensity}
	\item \textcolor{OrangeGreen}{A mix of 10\% green and 90\% orange}
	\item \textcolor{HTMLColor}{Defined with HTML code}
\end{itemize}

These can be used for any element that takes a colour as parameter, like a line.

\noindent {\color{TealBlue} \rule{\linewidth}{1mm}} % draws a line

\clearpage
\section{Table of contents}

\tableofcontents

% tocs don't include unnumbered sections, if one needs to be added, then the
% command \addcontentsline{toc}{section}{Unnumbered Section} can be used above
% that unnumbered section

\addcontentsline{toc}{section}{Include me please!} % example
\section*{Include me please!}

Table of contents don't include unnumbered sections. if one needs to be added, then the command \verb+\addcontentsline{toc}{section}{Unnumbered Section}+ can be used above that unnumbered section. For example, this unnumbered section is in the ToC.

\section*{I want to be left out!}

This unnumbered section is not in the ToC.

\clearpage

\section{Referencing with BibLaTeX}

Cite things like this: 
\cite{dirac}, 
\cite{einstein}, 
\cite{knuth-fa}, 
\cite{knuthwebsite}! You can change the referencing style in the preamble. 
Make sure your editor of choice is using Biber (or your backend of choice) 
as its default bibliography tool!

\printbibliography

\clearpage

\section{Input and include}

These commands allow to import content from other .tex files into the main 
documents, which makes things a lot easier when working on a bigger project.

\input{input} % file extension is optional

\include{include} % file extension should NOT be included

\end{document} % end of the document body